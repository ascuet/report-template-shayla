Some basic ways to manipulate text are \textit{italics}, \textbf{bold}, and \textcolor{blue}{colored text}. One can reference Figures (see Figure \ref{logo} for an example) as well as cite references which are defined in the \textit{references.bib} file.\cite{spectre,example-reference} 
\begin{center}
    \begin{figure}[ht] \label{logo}
\centering
\includegraphics[width=2 in]{figures/logo.jpg}
\centering
\end{figure} 
\end{center}
The \textit{Bibliography}, \textit{List of Figures} and \textit{List of Tables} are all automatically generated and references will be updated automatically as well. This means that if you've defined a citation but are not referencing it, it will not appear in the \textit{Bibliography}. This also means that any Figure / Table / Citations numbers are automatically updated as well. Numbering is done by order-of-appearance.
\section{Item, enumerate, and list}
One can create an itemized list:
(i) \begin{itemize}
    \item item a
    \item item b
    \item ...
\end{itemize}
(ii)
\begin{itemize}[label=\ding{212}]
\item First
\item Second
\end{itemize}
Or enumerate them:
(i) \begin{enumerate}
    \item item x
    \item item y
    \item ...
\end{enumerate}


(ii) \begin{enumerate}[label=\roman*]
  \item item x
  \item item y
  \item item ..
\end{enumerate}

(iii) \begin{enumerate}[label=\Roman*]
\item item x
  \item item y
  \item item ..
\end{enumerate}
or listed them

\begin{list}{$\circ$}{}  
\item A  
\item B  
\end{list}
Another example are showing below:
\begin{itemize}

\item[$\checkmark$] This will give a checkmark bullet.

\item[$\square$] This will give a hollow square bullet.

\item[$\blacksquare$] This will give a filled square bullet.

\item[$\bigstar$] This will give you a bigstar bullet.

\end{itemize}
We can use variables set in the \textit{main.tex} file to render values like our title (\doctitle) or supervisor names (\textbf{Supervisor}: \supervisor, \textbf{Co-supervisor}: \cosupervisor{}).

You may add following sections in this chapter.
\begin{itemize}
    \item objective
    \item motivation
    \item challenges
    \item Design approach
    \item contribution
    \item Organize of the thesis
    \item Summary
\end{itemize}
\section{Section, sub section, sub sub section, section without listing in the TOC}
\section{Section} \label{sec}
subsection will be under section.
\subsection{Sub section}\label{sub}
After subsection you can add "subsubsection"
\section{section/subsection without listing in TOC}
to do so you need to add "*" . For example
\subsection*{Section without listing}
This section won't be appear on TOC.