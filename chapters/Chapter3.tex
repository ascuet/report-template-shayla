\section{Algorithm \& Pseudo code \& Equations}
Package algorithm2e has been used here.
\subsection{Algorithm}
Canny edge detection is a popular multi-stage edge detection algorithm which was developed by John F. Canny in 1986 \cite{40}. 
\begin{algorithm}[ht]
\DontPrintSemicolon
\caption{Edge detection algorithm}
\SetAlgoLined
\KwInput{Gray Scale image}
\KwOutput{Image with thin line}
Calculate the gradient of all pixel in the X \& Y direction ($G_x$,$G_y$);

Compute magnitude and direction of each pixel( G, $\theta)$;\

Observe all nonmaximal value and reject them;\

Measure the high ($T_H$) and low($T_L$) thresholds using the histogram of the gradient magnitude of the image;\

\If{PixelValue>$T_H$}{
Consider the pixel as strong edge}
\ElseIf{($T_L$)< PixelValue <($T_H$)}
{Consider the pixel as weak edge}
\Else
{Reject}
\If{edge is strong edge}
{select the edge}
\ElseIf{Weak edge connected to any strong edge}
{select the edge}
\Else
{Reject}
\end{algorithm}
\subsection{Pseudo code}
\begin{algorithm}[ht]
\caption{Pseudocode}\label{euclid}
\begin{algorithmic}[1]
\Procedure{Psudocode}{}
\State $\textit{stringlen} \gets \text{length of }\textit{string}$
\State $i \gets \textit{patlen}$
\BState \emph{top}:
\If {$i > \textit{stringlen}$} \Return false
%\EndIf
\State $j \gets \textit{patlen}$
\BState \emph{loop}:
\If {$\textit{string}(i) = \textit{path}(j)$}
\State $j \gets j-1$.
\State $i \gets i-1$.
\State \textbf{goto} \emph{loop}.
\State \textbf{close};
%\EndIf
\State $i \gets i+\max(\textit{delta}_1(\textit{string}(i)),\textit{delta}_2(j))$.
\State \textbf{goto} \emph{top}.
\EndProcedure
\end{algorithmic}
\end{algorithm}
\subsection{Equations}
1st example:

\begin{algorithm}
\caption{Function}
  \begin{algorithmic}[1]
    \Function{Func\_name}{parameter}
      \State statement 1
      \State statement 2    
    \EndFunction
  \end{algorithmic}
\end{algorithm}

\begin{equation}
  h_{ij}= \dfrac{1}{2\pi\sigma^2} exp (-\dfrac{(i-(k+1))^2+(j-(k+1))^2}{2\sigma^2}; 1\le i, j\le (2k+1);
    \end{equation}
    2nd example:
    \begin{equation}
    \|Edge\_gradient(G)\|= \sqrt{I_x^2+I_y^2}
    Angle(\theta)= \tan^{-1}(\frac{I_y}{I_x})
\end{equation}
If you don't want to numbering any equation just use (\ no number) command.
\begin{equation}[H]
  SM1(x)=\begin{cases}
    Stops, & \text{if $\alpha=1$}.\\
    Moves, & \text{otherwise}.
  \end{cases}
  \nonumber
\end{equation}
\begin{equation}
  SM2(x)=\begin{cases}
  \mbox{\textit{Nod thrice}}, & \text{if $\beta=1$}.\\
   \mbox{\textit{Nods once}}, & \text{if $\beta=2$}\\
    \mbox{\textit{No movement}}, & \text{otherwise}.
  \end{cases}
  \nonumber
\end{equation}
\subsection{Matrix}[H]
 $$
 K_x=\begin{bmatrix} 
-1 & 0 & 1 \\
-2 & 0 & 2\\
-1 & 0 & 1
\end{bmatrix}
 K_y=\begin{bmatrix}
1 & 2 & 1 \\
0 & 0 & 0\\
-1 & -2 & -1
\end{bmatrix}
$$
\section{Flow chart}
\subsection{arrow and square}
\tikzstyle{ellip}=[draw,ellipse,fill=white!20,minimum height=2em]
\tikzstyle{rec}=[draw,rectangle,fill=white!20,text width=6em,text centered,minimum height=2em]
\tikzstyle{line}=[draw,-latex']
\begin{figure}[htbp]
\begin{center}
\begin{tikzpicture}[]

\node[ellip](step1) at (2,0){Start};
\node[rec] (step2) at (-1,-1) {Launch Speech to Text};
\node[rec] (step3) at (5,-1) {Launch Sign Language Keyboard};
\node[rec] (step4) at (-1,-3.3) {Speech to Sign Language Conversion with Help of Virtual Agent};
\node[rec] (step5) at (5,-3.5) {Convert Bangla Text};
\node[ellip] (step6) at (2,-4.7) {End};

\path[line] (step1) -| (step2);
\path[line] (step1) -| (step3);
\path[line] (step2) -- (step4);
\path[line] (step3) -- (step5);
\path[line] (step4) |- (step6);
\path[line] (step5) |- (step6);

\end{tikzpicture}
\end{center}
\caption{Overview of the proposed system}
\label{fig:first}
\end{figure}
\subsection{another example}
\tikzstyle{ellip}=[draw,ellipse,fill=white!20,minimum height=1em,minimum width=7.1em]
\tikzstyle{rec}=[draw,rectangle,fill=white!20,text width=6em,text centered,minimum height=2em]
\tikzstyle{line}=[draw,-latex']
\tikzstyle{line2}=[draw]
\tikzstyle{rec2}=[draw=white,rectangle,fill=white!20,text width=6em,text centered,minimum height=2em]
\begin{figure}[htbp]
\begin{center}
\begin{tikzpicture}[]

\node[rec] (step1) at (-1,0) {Speaking Bangla};
\node[rec] (step2) at (2,0) {Splitting Words from Sentence};
\node[rec] (step3) at (5,0) {Splitting suffixes};
\node[rec] (step4) at (5,-2) {Fetching Sign};
\node[rec] (step5) at (-1,-2) {Sign Language Output};
\node[rec2] (step7) at (2,-2){Raw folder containing animation};
\node[ellip] (step6) at (2,-1) {};

\path[line] (step1) -- (step2);
\path[line] (step2) -- (step3);
\path[line] (step3) -- (step4);
\path[line2] (step4) -- (5,-3);
\path[line] (5,-3) -| (step5);
\path[line2] (3.05,-1) -- (3.05,-2.8);
\path[line2] (3.05,-2.8) -- (0.95,-2.8);
\path[line2] (0.95,-2.8) -- (0.95,-1);
\draw[>= open triangle 45, <->](step4) -- (3.05,-2);
\end{tikzpicture}
\end{center}
\caption{Layout for showing the signs on the mobile screen}
\label{fig:second}
\end{figure}